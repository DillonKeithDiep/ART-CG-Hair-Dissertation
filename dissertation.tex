% The document class supplies options to control rendering of some standard
% features in the result.  The goal is for uniform style, so some attention 
% to detail is *vital* with all fields.  Each field (i.e., text inside the
% curly braces below, so the MEng text inside {MEng} for instance) should 
% take into account the following:
%
% - author name       should be formatted as "FirstName LastName"
%   (not "Initial LastName" for example),
% - supervisor name   should be formatted as "Title FirstName LastName"
%   (where Title is "Dr." or "Prof." for example),
% - degree programme  should be "BSc", "MEng", "MSci", "MSc" or "PhD",
% - dissertation title should be correctly capitalised (plus you can have
%   an optional sub-title if appropriate, or leave this field blank),
% - dissertation type should be formatted as one of the following:
%   * for the MEng degree programme either "enterprise" or "research" to
%     reflect the stream,
%   * for the MSc  degree programme "$X/Y/Z$" for a project deemed to be
%     X%, Y% and Z% of type I, II and III.
% - year              should be formatted as a 4-digit year of submission
%   (so 2014 rather than the accademic year, say 2013/14 say).

\documentclass[ % the name of the author
                    author={Dillon Keith Diep},
                % the name of the supervisor
                supervisor={Dr. Carl Henrik Ek},
                % the degree programme
                    degree={MEng},
                % the dissertation    title (which cannot be blank)
                     title={Assisted Content Generation for 3D Hair Geometry},
                % the dissertation subtitle (which can    be blank)
                  subtitle={[INCOMPLETE DRAFT, CONTAINS NOTES FROM RESEARCH]},
                % the dissertation     type
                      type={Research},
                % the year of submission
                      year={2014} ]{dissertation}

\begin{document}

% =============================================================================

% This section simply introduces the structural guidelines.  It can clearly
% be deleted (or commented out) if you use the file as a template for your
% own dissertation: everything following it is in the correct order to use 
% as is.

% =============================================================================

% This macro creates the standard UoB title page by using information drawn
% from the document class (meaning it is vital you select the correct degree 
% title and so on).

\maketitle

% After the title page (which is a special case in that it is not numbered)
% comes the front matter or preliminaries; this macro signals the start of
% such content, meaning the pages are numbered with Roman numerals.

\frontmatter

% This macro creates the standard UoB declaration; on the printed hard-copy,
% this must be physically signed by the author in the space indicated.

\makedecl

% LaTeX automatically generates a table of contents, plus associated lists 
% of figures, tables and algorithms.  The former is a compulsory part of the
% dissertation, but if you do not require the latter they can be suppressed
% by simply commenting out the associated macro.

\tableofcontents
\listoffigures
\listoftables
\listofalgorithms
\lstlistoflistings

% The following sections are part of the front matter, but are not generated
% automatically by LaTeX; the use of \chapter* means they are not numbered.

% -----------------------------------------------------------------------------

\chapter*{Executive Summary}

{\bf A compulsory section, of at most $1$ page} 
\vspace{1cm} 

\noindent
The research hypothesis of this study is that probabilistic principal component analysis with the
Gaussian Process Latent Variable Model is applicable for improving the creative production workflow of complex 3D geometry such as hair structures of humanoids.

The topic of this thesis explores the concept of assisted content generation by machine learning for the production of 3D hair geometry. The production of 3D virtual worlds is a time-consuming and costly process that also demand expert knowledge. 3D assets encompass a vast range of applications, ranging from simulations and research, to contributing towards the functioning of many businesses. The production of 3D assets also plays a pivotal role in engineering design, and the provisioning of entertainment. One particular task is the creation of 3D hair geometry for humanoid characters. The production of 3D hair is arduous as hair structure is a complex system containing much interdependence between components.

Machine learning applications typically use large data sets for training on problems that often have a concise answer for a given prediction. The application of machine learning to enhance production for creative work is an exciting field that tackles new challenges - not only could it have a notable impact on the economy, artistic products tend to have small sets of data available and evaluation of quality is subjective. Given the same input, acceptable solutions can vary significantly. The mentioned peculiarities of applying machine learning for 3D mesh data defines a unique field of problems to investigate.

Existing tools for 3D modelling have remained mostly static in the paradigm of approach over the past few decades. Automation through methods such as procedural generation can produce output much faster, but the lack of control over the final result makes it less desirable than traditional methods of 3D modelling. The focus of this project is to formulate a revolutionary framework that improves the workflow of producing 3D hair geometry.

\begin{quote}
\noindent
\begin{itemize}
\item Resolved the alignment problem by coming up with a generative model for approximating hair structure
\item Demonstrated the use of GPLVM on various sizes of data set
\item Implemented an add-on package for a 3D production program, Blender
	\begin{itemize}
		\item This implementation creates guiding splines that are useful for generating geometry
		\item Appropriate for small training set that is practical for content creators
		\item Real-time performance that matches current processing time required by traditional tools
	\end{itemize}
\item [TODO]: Evaluating kernels
\item [TODO]: Analysis and Evaluation
\end{itemize}
\end{quote}

% -----------------------------------------------------------------------------

\chapter*{Supporting Technologies}

{\bf A compulsory section, of at most $1$ page}
\vspace{1cm} 

\noindent
This section should present a detailed summary, in bullet point form, 
of any third-party resources (e.g., hardware and software components) 
used during the project.  Use of such resources is always perfectly 
acceptable: the goal of this section is simply to be clear about how
and where they are used, so that a clear assessment of your work can
result.  The content can focus on the project topic itself (rather,
for example, than including ``I used \mbox{\LaTeX} to prepare my 
dissertation''); an example is as follows:

\begin{quote}
\noindent
\begin{itemize}
\item I used the Java {\tt BigInteger} class to support my implementation 
      of RSA.
\item I used a parts of the OpenCV computer vision library to capture 
      images from a camera, and for various standard operations (e.g., 
      threshold, edge detection).
\item I used an FPGA device supplied by the Department, and altered it 
      to support an open-source UART core obtained from 
      \url{http://opencores.org/}.
\item The web-interface component of my system was implemented by 
      extending the open-source WordPress software available from
      \url{http://wordpress.org/}.
\end{itemize}
\end{quote}

% -----------------------------------------------------------------------------

\chapter*{Notation and Acronyms}

{\bf An optional section, of roughly $1$ or $2$ pages}
\vspace{1cm} 

\noindent
Any well written document will introduce notation and acronyms before
their use, {\em even if} they are standard in some way: this ensures 
any reader can understand the resulting self-contained content.  

Said introduction can exist within the dissertation itself, wherever 
that is appropriate.  For an acronym, this is typically achieved at 
the first point of use via ``Advanced Encryption Standard (AES)'' or 
similar, noting the capitalisation of relevant letters.  However, it 
can be useful to include an additional, dedicated list at the start 
of the dissertation; the advantage of doing so is that you cannot 
mistakenly use an acronym before defining it.  A limited example is 
as follows:

\begin{quote}
\noindent
\begin{tabular}{lcl}
AES                 &:     & Advanced Encryption Standard                                         \\
DES                 &:     & Data Encryption Standard                                             \\
                    &\vdots&                                                                      \\
${\mathcal H}( x )$ &:     & the Hamming weight of $x$                                            \\
${\mathbb  F}_q$    &:     & a finite field with $q$ elements                                     \\
$x_i$               &:     & the $i$-th bit of some binary sequence $x$, st. $x_i \in \{ 0, 1 \}$ \\
\end{tabular}
\end{quote}

% -----------------------------------------------------------------------------

\chapter*{Acknowledgements}

{\bf An optional section, of at most $1$ page}
\vspace{1cm} 

\noindent
It is common practice (although totally optional) to acknowledge any
third-party advice, contribution or influence you have found useful
during your work.  Examples include support from friends or family, 
the input of your Supervisor and/or Advisor, external organisations 
or persons who  have supplied resources of some kind (e.g., funding, 
advice or time), and so on.

% =============================================================================

% After the front matter comes a number of chapters; under each chapter,
% sections, subsections and even subsubsections are permissible.  The
% pages in this part are numbered with Arabic numerals.  Note that:
%
% - A reference point can be marked using \label{XXX}, and then later
%   referred to via \ref{XXX}; for example Chapter\ref{chap:context}.
% - The chapters are presented here in one file; this can become hard
%   to manage.  An alternative is to save the content in seprate files
%   the use \input{XXX} to import it, which acts like the #include
%   directive in C.

\mainmatter

% -----------------------------------------------------------------------------

\chapter{Contextual Background}
\label{chap:context}

\section{Topic Background}
\noindent
3D Representations in Computer Graphics
In computer graphics, 3D objects are represented in many forms. 3D scanners capture raw data in various forms such as point clouds, range images, and voxels. A point cloud is a collection of 3D points, often used in computer vision.  A range image maps pixels of a depth image to a set of points in the scene. Voxels are units of cubes that define the volume of objects, it has applications in many fields including medicine where voxels are used to visualise the results of MRI scans of patients.\cite{mri}

In a production environment it is simpler to define geometry as opposed to capturing examples. The most common representation used for CG production are polygonal meshes, data that contains information of vertices, edges, and faces. Topology is the organisation of the components that define the mesh geometry. Two surfaces with the same appearance could have different topologies. Where precision is concerned, parametric definitions are used for industries such as CAD. Every representation has its advantages depending on the use case. It is possible to convert between representations, but data loss may be incurred. Properties that make polygon meshes desirable include being efficient rendering, simple to define, expressive enough to capture geometry required, and works well with established techniques such as UV texture mapping and deforming algorithms. The rendering pipeline often converts meshes to tri-faces (faces constructed by three edges) as an optimisation process, but best practice for 3D artists is to maintain a topology of quad-faces which are easier to organise and conforms better with editing tools and algorithms.

%** Image of representations

\section{Production of 3D Hair Geometry}
\noindent
On average, a human is born with about 100,000 scalp hair follicles. It is very expensive to render and animate physically correct hair, but creative liberties have been taken to approximate or stylize 3D hair such that it is both acceptable aesthetically and feasible in terms of performance. This study considers modelling of hair geometry, the motion of hair is assumed to be its default resting pose.

In recent years, impressive 3D hair solutions for real-time simulation of realistic hair and fur, such as Nvidia HairWorks and AMD’s TressFX has emerged. These solutions, however, have limited application in comparison to their traditional counterpart of polygonal hair. It is often the case that texture-mapped polygonal hair is used as a fallback for when the advanced simulation fails. Realism is not necessarily always desirable, polygon hair can flexibly represent different art styles. In some cases, a blend of multiple representations are used to balance between cost and quality. 3D hair in movies with large budget can afford to render hair with much higher fidelity for important characters, but might use efficient variants for scenarios such as crowd simulation. Ultimately, representation of virtual hair generally follows a structure of splines with control points that define the overall organisation of strands or segments.


[Image of Hair Geometry]

\section{Procedural Generation and Automated Production}
\noindent
Procedural generation techniques has been successfully used for terrains and city modelling. Methods such as the Lindenmayer system are used for foliage generation. Fractals with simulated noise can create structure that resemble patterns observed in nature. Procedural techniques, however, are seldom used for modelling objects with specific features. It is difficult to control the output of procedurally generated content without heavily restricting its capabilities.
\cite{procedural1}
\cite{procedural2}

Machine Learning in 3D Production
Machine learning methods can use existing production as training input to train models that are more concise and versatile than predetermined procedural approaches. Research in the past has used machine learning to help non-artists drawing 2D images. Gaussian processes enabled stylised inverse kinematics that deliver simple controls for complicated animation by learning the posing of joints. 
\cite{MLMethods}

\section{Motivation and Significance}
\noindent
State of the art 3D production software such as AutoDesk Maya, 3DS Max, and Blender are advanced programs with sophisticated list of features. That said, such programs have extremely convoluted user interfaces, even the most experienced professionals do not recognise each and every tool available. The most versatile tools are generally the most basic that perform atomic changes as they are applicable in every scenario. Examples include selection of primitives such as vertices, edges, or faces and performing translation, rotation, and scaling. Sculpting tools moves many data points simultaneously, they are popular for defining organic surfaces now that modern machines are sufficiently powerful. Experienced artists might search for an existing base mesh that is similar to start from, but it is not always the case that such a base mesh exists - there are also concerns for quality, such as unorganised topology. As the geometry becomes more detailed and well-defined, each alteration makes less impact and the space of sensible edits becomes smaller. The design and production of 3D geometry remains a slow and delicate process.

Virtual hair creation is a necessity for characters of CG movies and video games that are embedded within culture both economically and as entertainment. Specialised artists learn to be proficient with the design of hair, variety of styles, and techniques for creating them. Hair geometry is much more concentrated than other types, containing many data points that are tedious to edit. Soft selection and sculpting tools are good enough for defining the structure but maintaining topology and issues such as overlapping surfaces are still problematic. Learning the relation of hair structure allows the potential of discovering new hairstyles.  It can also be used as a mean of rapidly generating initial base geometry that fits the target output better than existing geometry found. Generative methods could ensure a level of quality, clean topology that fits established specifications. Assisted content generation using machine learning provides a convenient, non-intrusive and intuitive method for rapidly generating new hair geometry from existing data.

%**Move to evaluation
The application of machine-learning based tools could enhance the workflow of professional users and improve the experience for non-expert consumers. Such tools integrate into the production environment to improve the efficiency of acquiring initial base geometry and visually compare designs during pre-production. Non-expert users receive the ability to produce 3D geometry without requiring to learn the intrinsics of traditional 3D modelling software. The rise in popularity for augmented reality and 3D printing inspires the development of generative tools that are intuitive and simplistic to use. Applications that allow users to create their personal content could also integrate machine-learning based tools to prevent inappropriate or undesirable creation from being produced while providing options that surpass existing alternatives. An example would be avatar creation for many applications and video games. A space of reasonable options generated from predefined outputs by the developers will allow users to interpolate between sensible configurations, providing an excellent level of customisation while adhering to defined constraints.

\section{Challenges}
\noindent
This study faces a number of challenges. First of all, 3D meshes are difficult to compare. The training data in its raw form will have varying dimensions. Meshes can be viewed as samples of the true geometry, thus meshes that represent the same object could differ drastically in number of data points depending on its level of detail. Typical feature extraction methods do not work well on meshes as artistic products are sensitive to data loss - any change could affect the final result drastically.

Another problem encountered is the lack of training data. Typical machine learning solutions use huge data sets in the order of hundreds of thousands for training, but for 3D meshes the expected size of readily available training data is much smaller. Public repositories of 3D polygonal hair are generally around a few thousand in size. Studios that store and organise past production could likely match the size of public repositories, depending on the size of the company. Independent artists that keep their production will range in the hundreds.
\cite{tsr}

The application of machine learning methods must also account for subjectivity of evaluating artistic assets. The range of acceptable solutions is ambiguous, likened to how hair styles of characters can change drastically during the design phase, determining the threshold of acceptable solutions will be in itself a challenge to resolve.

As mentioned previously, 3D meshes are delicate and can easily be invalidated from small changes. Thus, reparations are required to ensure that the output of trained models are acceptable.

In a production environment, the time required for a technique to return observable result directly affects throughput. For practical efficacy of assisted content generation, usage of our methods should be reasonably effective.

\section{Central Objectives}
\noindent
The aim of this study is …
\begin{itemize}
\item Resolving the alignment problem of 3D data by standardisation.
\item Explore the application of GPLVM for 3D hair geometry in a production pipeline.
\item Investigate the use of latent variables for identifying stylistic properties of 3D hair geometry.
\item Demonstrate the use of non-linear manifold to generate new hairstyles from training data.
\item Enable an intuitive method for non-experts to create 3D hair geometry.
\item Observable output demand performance close to real-time for practical use.
\end{itemize}


% -----------------------------------------------------------------------------

\chapter{Technical Background}
\label{chap:technical}

\section{Formal Definition of 3D Polygon Mesh Representation}
%\cite{polyhedron}The winged-edge polyhedron representation proposed by Baumgart in 1972 introduced a mesh data structure to represent 3D geometry using vertices, edges, and faces. \cite{wingededge}

Polygon mesh representation of 3D surfaces are composed of vertices, edges, and faces. Let polygon mesh $P = (\mathbf{V, E, F})$, where $\mathbf{V, E, F}$ represents the set of vertices, edges, and faces respectively. In practice, polygonal meshes contain more components that affect surface appearance such as texture coordinates and vertex normals, however, the components described are sufficient for geometric processing.

\subsection{Mesh Vertices}
A mesh vertex $v$ is a 3D point of the form $\forall (x, y, z) \in \Re, v = (x, y, z)$.
The set of vertices is a point cloud representation of the geometry. 

\subsection{Edges}
An edge $e$ is an unordered pair that connects two vertices. Formally, it is described in the form $\forall (v_1, v_2) \in \mathbf{V}, e = \{v_1, v_2\}$. Vertices connected by edges form a wireframe of the geometry. 

\subsection{Polygon Faces}
A polygon face can be formed from an arbitrary number of vertices $\forall (v_1, v_2,...,v_n) \in \mathbf{V}, f_n = (v_1, v_2,..., v_n)$, however, in this context we are only concerned with tri-faces $\forall (v_1, v_2, v_3) \in \mathbf{V}, f_3 = (v_1, v_2, v_3)$ and quad-faces $\forall (v_1, v_2, v_3, v_4) \in \mathbf{V}, f_4 = (v_1, v_2, v_3, v_4)$. Faces describe the geometric surface of an object. 

\subsection{Edge Loops}
3D programs often allow edge loop selection which are useful properties of the geometry. An edge loop is defined (on blender) as a set of connected edges that either forms a loop or the end vertices are poles (vertices that do not have edges). Edge loops are useful for extracting more information on the structure of the mesh.
\cite{edgeloops}

\section{Principal Component Analysis}
\subsection{PCA [Temp Main]}
Intro:
In multivariate analysis, principal component analysis (PCA) is a statistical technique used to perform dimensionality reduction.\cite{pca2002} It was originally introduced by Pearson\cite{pca1901}, and independently developed by Hotelling\cite{pca1933}, where the standard algebraic derivation of PCA was presented in terms of a standardized linear projection.

Intuition:
Consider variables of hair structure: location, orientation, length, color. There is some correlation between the variables, this is measured as covariance. PCA represents these observed variables as a combination that is independent, so location, orientation, length may be combined but not color. This is captured by selecting orthonormal axes that retain the most variance.

Derivation1:
Given a set of $n$ observed $d$-dimensional data vectors $\mathbf{X}=\{\mathbf{x}_1,\mathbf{x}_2,...,\mathbf{x}_n\}$, $\mathbf{x}_i = \{x_1,x_2,...,x_d\}$, principal component analysis can be solved as an eigenvalue decomposition problem to determine a selected size of $m$ principal components, where $\mathbf{Y}=\{\mathbf{y}_1,\mathbf{y}_2,...,\mathbf{y}_m\}$, $\mathbf{y}=\{y_1,y_2,...,y_d\}$. 

\subsection{PCA}
The most common derivation of PCA is in terms of a standardized linear projection which maximises the variance in the projected space (Hotelling, 19333). For a set of observed d-dimensional data vectors $\{\mathbf{t}_n\}$, $n\in {1,...,N}$, the q principal axes $\mathbf{w}_j, j\in \{1,...,q\}$, are those orthonormal axes onto which the retained variance under projection is maximal. It can be shown that the vectors $\mathbf{w}_j$ are given by the q dominant eigenvectors (those with the largest associated eigenvalues $\lambda_j$) of the sample covariance matrix $\mathbf{Sw}_j=\lambda_j\mathbf{w}_j$. The q principal components of the observed vector $\mathbf{t}_n$ are given by the vector $\mathbf{x}_n=\mathbf{W}^T(\mathbf{t}_n-\mathbf{\bar{t}})$, where $\mathbf{W=(w_1,w_2,...,w}_q)$. The variables $x_j$ are then uncorrelated such that the covariance matrix $\sum_n\mathbf{x}_n\mathbf{x}^T_n /N$ is diagonal with elements $\lambda_j$.

A complementary property of PCA is that of all orthogonal linear projections $\mathbf{x}_n=\mathbf{W}^T(\mathbf{t}_n-\mathbf{\bar{t}})$, the principal component projection minimises the squared reconstruction error $\sum_n||\mathbf{t}_n-\mathbf{\hat{t}}_n||^2$, where the optimal linear reconstruction of $\mathbf{t}_n$ is given by $\mathbf{\hat{t}=Wx}_n+\mathbf{\bar{t}}$.

\subsection{Single Value Decomposition of X}
PCA can be seen as SVD of X, $\mathbf{X=U\Sigma V^T}$.

\vspace{5cm}
\section{Probabilistic Principal Component Analysis}
PPCA Tipping and Bishop 1999
\subsection{Latent Variable Models and Factor Analysis}
A latent variable model transforms a set of $n$ $d$-dimensional observed variables, $\mathbf{Y}\in\Re^{n \times d}$, to a set of $q$-dimensional latent (unobserved) variables, $\mathbf{X}\in\Re^{n \times q}$. Latent variables are parsimonious, it is generally the case that $q \ll d$, explaining the original data with fewer variables. A notable latent variable model is that of factor analysis, one that assumes linearity in relation of the observed data set.
\begin{equation} \label{ppca:fa}
	\mathbf{Y=WX+\mu+\epsilon}
\end{equation}
$\mathbf{W}$ represents a matrix that specifies the relation between the observed data-space with the latent-space.
The parameter $\mu$ allows for non-zero mean, and the $\epsilon$ parameter represents noise within the model.

\subsection{Relation of Factor Analysis and PCA}

Standard PCA does not differentiate between covariance and variance. FA latent variables explain the correlations between the variables observed, and $\epsilon_i$ represents variability for $\mathbf{y_i}$. The maximum-likelihood estimates of the columns of $\mathbf{W}$ will thus generally \textit{not} correspond to the principal subspace of the observed data.

A special case is the isotropic error model. Constraining the residual variances $\psi_i=\sigma^2$ to be equal, the maximum-likelihood is equivalent to a least-squares criterion, and a principal component solution emerges.
The observation covariance model $\mathbf{WW}^T+\sigma^2\mathbf{I}$ can be made exact (assuming correct choice of q), and both $\mathbf{W}$ and $\sigma^2$ may then be determined analytically through eigen-decomposition of S (Anderson 1963; Basilevsky 1994, pp.361-363).

Lawley (1953) and Anderson and Rubin (1956) show that stationary points of the likelihood function occur when $\mathbf{W}$ is a matrix whose columns are scaled eigenvectors of the sample covariance matrix $\mathbf{S}$, and $\sigma^2$ is the average variance in the discarded dimensions. Falls short of showing that the principal eigenvectors represent the global maximum of the likelihood.

The maximum likelihood estimators $\mathbf{W}_{ML}$ and $\sigma^2_{ML}$ of factor analysis when using isotropic Gaussian distribution for $\epsilon$ corresponds to principal component analysis.

PCA may be viewed as a maximum likelihood procedure based on a probability density model of the observed data (Lawley 1953, Anderson and Robin 1956).

Tipping and Bishop extend Lawley and Anderson and Rubin to show how PCA may be viewed as a maximum likelihood procedure based on a probability density model of the observed data.

\subsection{The probability model}
Let parameter $\epsilon$ of equation \ref{ppca:fa} be modelled as an isotropic, spherical Gaussian distribution $\mathcal{N}(\mathbf{0},\sigma^2\mathbf{I})$. We obtain the conditional probability distribution:
\begin{equation} \label{ppca:conditional}
	p(\mathbf{Y|X})\sim\mathcal{N}(\mathbf{WX+\mu},\sigma^2\mathbf{I}). 
\end{equation}

The marginal distribution over the latent variables are Gaussian, defined by $\mathbf{X\sim\mathcal{N}(0,I)}$. The marginal distribution for the observed data $\mathbf{Y}$ is obtained by integrating out the latent variables,
$$\mathbf{Y}\sim\mathcal{N}\mathbf{(\mu,C)}.$$
The observation covariance model is $\mathbf{C=WW}^T+\sigma^2\mathbf{I}$. The corresponding log-likelihood is
\begin{equation} \label{ppca:loglikelihood}
	\mathcal{L}=\frac{n}{2}(d ln(2\pi)+ln|\mathbf{C}|+tr(\mathbf{C^{-1}S}))
\end{equation}
where
$$\mathbf{S}=\frac{1}{n}\sum^n_{i=1}(\mathbf{y}_i-\mu)(\mathbf{y}_i-\mu)^T$$
The conditional distribution of the latent variables $\mathbf{x}$ given the observed $\mathbf{y}$, which may be calculated by using Bayes theorem is:
$$p(\mathbf{x|y)\sim\mathcal{N}(M^{-1}W}^T\mathbf{(y-\mu),\sigma^2M^{-1})}$$
where $\mathbf{M=W}^T\mathbf{W+\sigma^2I}$, and $\mathbf{M}$ is a $q \times q$ matrix.

\subsection{Principal Subspace}
Given that the observation covariance model, $\mathbf{C}=\mathbf{WW}^T\sigma^2\mathbf{I}$, the likelihood (\ref{ppca:loglikelihood}) is maximised when
\begin{equation} \label{ppca:ml}
	\mathbf{W}_{ML}=\mathbf{U}_q(\Lambda_q-\sigma^2\mathbf{I})^{\frac{1}{2}}\mathbf{R}
\end{equation}
where the $q$ column vectors in the $d \times q$ matrix $\mathbf{U}_q$ are the principal eigenvectors of $\mathbf{S}$, with corresponding eigenvalues $\lambda_1,...,\lambda_q$ in the $q \times q$ diagonal matrix $\Lambda_q$, and $\mathbf{R}$ is an arbitrary $q \times q$ orthogonal rotation matrix. From equation \ref{ppca:ml}, the latent variable model defined by equation \ref{ppca:fa} effects a mapping from the latent space into the \textit{principal subspace} of the observed data.

\subsection{Maximum-Likelihood PCA}
 $\Lambda_q=[ \lambda_1,...,\lambda_q ]$.

The matrix $\mathbf{U_q}$ may contain any of the eigenvectors of $\mathbf{S}$, so to identify those which maximise the likelihood, the expression for $\mathbf{W}$ in (15) is substituted into the log-likelihood function (4) to give 
$$\mathcal{L}=-\frac{n}{2} ( d\ln(2\pi)+\sum^{q'}_{j=1}\ln(\lambda_j)+\frac{1}{\sigma^2}\sum^d_{j=q+1}\lambda_j+(d-q')\ln\sigma^2+q' )$$

Where $q'$ is the number of non-zero $l_j$, $\lambda_1,...,\lambda_{q'}$ are eigenvalues corresponding to the eigenvectors 'retained' in W, and $\lambda_{q'+1,...,\lambda_d}$ are those 'discarded'. Maximising (17, L) with respect to $\sigma^2$ gives
$$\sigma^2=\frac{1}{d-q}\sum^d_{j=q'+1}\lambda_j$$
so
$$\mathcal{L}=-\frac{n}{2}( \sum^{q'}_{j=1}\ln(\lambda_j)+(d-q')\ln( \frac{1}{d-q'}\sum^d_{j=q'+1}\lambda_j ) +d\ln(2\pi)+d )$$

To find the maximum of (19, L), with respect to the choice of eigenvectors and eigenvalues to retain in $\mathbf{W}$, and those to discard.
By exploiting the constancy of the sum of all eigenvalues, the condition for maximisation of the likelihood can be expressed equivalently as minimisation of the quantity
$$E=\ln(\frac{1}{d-q'}\sum^d_{j=q'+1}\lambda_j)-\frac{1}{d-q'}\sum^d_{j=q'+1}\ln(\lambda_j)$$

L is maximised with respect to $q'$ when there are fewest terms in the sums in (20, E) which occurs when $q'=q$ and therefore no $l_j$ is zero. Furthermore, $L$ is minimised when $\mathbf{W=0}$, which is equivalent to the case where $q'=0$.

\subsubsection{The Stationary Points of the Log-Likelihood}
The gradient of the log-likelihood (\ref{ppca:loglikelihood}) with respect to $\mathbf{W}$ is obtained from matrix differentiation (Kranowsi and Marriott 1994):
$$\frac{\delta\mathcal{L}}{\delta\mathbf{W}}=N(\mathbf{C^{-1}SC^{-1}W-C^{-1}W}),$$
at the stationary points
\begin{equation} \label{ppca:stationary}
	\mathbf{SC^{-1}W=W}.
\end{equation}
There are three possible classes of solutions to equation \ref{ppca:stationary}. Trivially, $\mathbf{W=0}$ is a minimum of the log-likelihood. The case when the covariance model are identical, $\mathbf{C=S}$, $\mathbf{W}$ has a known solution as $\mathbf{WW^T=S-\sigma I}$, thus $\mathbf{W=U(\Lambda-\sigma^2 I)^\frac{1}{2}R}$, where $\mathbf{U}$ is a square matrix whose columns are the eigenvectors of $\mathbf{S}$, with $\mathbf{\Lambda}$ the corresponding diagonal matrix of eigenvalues, and $\mathbf{R}$ is an arbitrary orthogonal matrix.

The third case, where $\mathbf{SC^{-1}W=W}$, but $\mathbf{W \ne 0}$  and $\mathbf{C \ne S}$. To find these, express the parameter matrix in terms of its singular value decomposition:
\begin{equation} \label{ppca:svd1}
	\mathbf{W=ULV}^T
\end{equation}

As Tipping and Bishop noted, it turns out that all potential solutions for $\mathbf{W}$ may be written as $\mathbf{W=U_q(K_q-\sigma^2I)^\frac{1}{2}R}$.


\section{Gaussian Process Latent Variable Model}

\subsection{Probabilistic PCA}
Consider a set of $n$ centred $d$-dimensional data $\mathbf{Y=\{y_1,y_2,...,y_n\}}$.
For each observed data point, $1 \leq i \leq n $, there is an associated $q$-dimensional latent variable $\mathbf{x_i}$.
The original data can be represented in terms of the latent variable with noise value,
$$\mathbf{y_i=Wx_i+\epsilon_i}.$$
The matrix $\mathbf{W \in \Re^{d \times q}}$ represents the linear relationship between the latent-space with the  data-space. The noise values, $\epsilon_n \in \Re^{d \times 1}$, are sampled as independent spherical Gaussian distributions defined by $p(\epsilon_i)\sim\mathcal{N}(\mathbf{0, \beta^{-1}I})$.

Using properties of Gaussian distribution, the likelihood of a data point is thus
\begin{equation} \label{ppca:likelihood}
p(\mathbf{y_i|x_i,W,\beta)\sim\mathcal{N}(y_i|Wx_i,\beta^{-1}I)}.
\end{equation}
Integrating over the latent variables gives the marginal likelihood,
$$p(\mathbf{y_i|W,\beta})=\int p(\mathbf{y_i|x_i,W,\beta})p(\mathbf{x_i})d\mathbf{x_i}.$$
The prior of probabilistic PCA is modelled as a unit Gaussian distribution $p(\mathbf{x_i})\sim\mathcal{N}(\mathbf{x_i|0,I})$.
Marginalisation of the integral obtains the marginal likelihood of each data point as
$$p(\mathbf{y_i|W,W,\beta})\sim\mathcal{N}(\mathbf{y_i|0,WW^T+\beta^{-1}I}).$$
Assuming that the data points are independent, the likelihood of the full data set is the product of each marginal likelihood,
$$p(\mathbf{Y|W,\beta})=\prod^N_{i=1} p(\mathbf{y_i|W,\beta}).$$

\subsection{Dual Probabilistic PCA}
Lawrence introduced the dual probabilistic PCA (DPPCA) as an alternative solution to PPCA that is equivalent. Instead of optimising parameters and marginalising latent variables, the dual approach marginalises the parameters, $\mathbf{W}$, and optimises with respect to latent variables, $\mathbf{X}$.

A conjugate prior to the likelihood of PPCA (\ref{ppca:likelihood}) is a spherical Gaussian distribution
$$p(\mathbf{W})=\prod^d_{i=1}\mathcal{N}(\mathbf{w}_i|\mathbf{0,I}),$$
Marginalisation of both $\mathbf{W}$ and $\mathbf{X}$ is intractable. Marginalisation of W is simple with the conjugate prior. The marginalised likelihood of $\mathbf{W}$ is
$$p(\mathbf{Y|X},\beta)=\prod^d_{i=1}p(\mathbf{y}_{:,i}|\mathbf{X},\beta)$$
where $\mathbf{y}_{:,i}$ represents the $i$th column of $\mathbf{Y}$ and
$$p(\mathbf{y}_{:,i}|\mathbf{X},\beta)=\mathcal{N}(\mathbf{y}_{:,i}|\mathbf{0,XX}^T+\beta^{-1}\mathbf{I}).$$

The objective function is the log-likelihood
\begin{equation} \label{dppca:loglikelihood}
	L=-\frac{dn}{2}ln2\pi-\frac{d}{2}ln|\mathbf{K}|-\frac{1}{2}tr(\mathbf{K^{-1}YY}^T)
\end{equation}
where $$\mathbf{K=XX}^T+\beta^{-1}I.$$

The gradients of the log-likelihood (\ref{dppca:loglikelihood}) with respect to $\mathbf{X}$ may be found (Magnus and Neudecker, 1999) as
$$\frac{\sigma L}{\sigma \mathbf{X}}=\mathbf{K^{-1}YY}^T\mathbf{K^{-1}X}-D\mathbf{K^{-1}X},$$
a fixed point where the gradients are zero is then given by 
$$\frac{1}{D}\mathbf{YY}^T\mathbf{K^{-1}X=X}.$$
The values for $\mathbf{X}$ which maximise the likelihood are given by 
$$\mathbf{X=ULV}^T$$
where $\mathbf{U}$ is an $N\times q$ matrix whose columns are the first eigenvectors of $\mathbf{YY}^T,\mathbf{L}$ is a $q\times q$ diagonal matrix whose $jth$ element is $l_j=(\lambda_j-\frac{1}{\beta})^{-\frac{1}{2}}$ where $\lambda_j$ is the eigenvalue associated with the $jth$ eigenvector $D^{-1}\mathbf{YY}^T$ and $V$ is an arbitrary $q\times q$ rotation matrix. Assuming that the eigenvalues are ordered according to magnitude with largest first. The eigenvalue problem developed is equivalent to that solved in PCA.

\subsection{Gaussian Processes}
Gaussian processes (O'Hagan, 1992; Williams, 1998) are a class of probabilistic models which specify distribution over funciton spaces.
GPs work over infinite dimensions, they are not functions as processes approximate over a finite range.
A GP first requires specifying a Gaussian process prior, parametrised by a mean and covariance.

A simple GP prior over the space of functions that are linear but corrupted by Gaussian noise of variance $\beta^{-1}\mathbf{I}$ is
$$k(\mathbf{x}_i,\mathbf{x}_j)=\mathbf{x}^T_i\mathbf{x}_j+\beta^{-1}\delta_{ij}$$
where $\mathbf{x}_i$ and $\mathbf{x}_j$ are vectors from the space of inputs to the function and $\sigma_1{ij}$ is the Knronecker delta. If these inputs were taken from our embedding matrix, $\mathbf{X}$, and the covariance function was evaluated at each of the $N$ points, we would recover the covariance matrix of the form
$$\mathbf{K=XX}^T+\beta^{-1}\mathbf{I}$$
where the element at $ith$ row and $jth$ column ok $\mathbf{K}$ is given by the simple GP prior. This is recognised as the covariance associated with each factor of the marginal likelihood for dual probabilistic PCA. The marginal likelihood for dual probabilistic PCA is therefore a product of D independent Gaussian processes. In PCA we are optimising parameters and input positions of a Gaussian process prior distribution where the (linear) covariance function for each dimension is given by $\mathbf{K}$

\subsection{GP-LVM}

DPPCA points to a new class of models which consist of Gaussian process mappings from a latent space, $\mathbf{X}$, to an observed data-space, $\mathbf{Y}$. DPPCA is the special case where the output dimensions are \textit{a priori} assumed to be linear, independent and identically distributed. However, each of these assumptions can be infringed to obtain new probabilistic models. Independence can be broken by allowing an arbitrary rotation on the data matrix $\mathbf{Y}$, the 'identically distributed' assumption can be broken by allowing different covariance functions for each output dimension. By replacing the inner product kernel with a covariance function that allows for non-linear functions we obtain a non-linear latent variable model. Due to the close relationship with the linear model, which has an intepretation as probabilistic PCA, such a model can be interpreted as a non-linear probabilistic version of PCA.

\section{Bayesian Gaussian Process Latent Variable Model}
The Bayesian Gaussian Process Latent Variable Model (Bayesian GP-LVM) extends the GP-LVM by variationally integrating out the input variables of the Gaussian process to compute a lower bound on the exact marginal likelihood of the nonlinear latent variable model, thus becoming robust to overfitting. \cite{bgplvm}

To apply variational Bayes to GP-LVM, need to approximately integrate out latent/input variables that appear nonlinearly in the inverse kernel matrix of the GP model. Standard mean field variational methodologies is not tractable.

Recall that the marginalised likelihood of the GPLVM is
$$p(\mathbf{Y|X},\beta)=\prod^d_{i=1}p(\mathbf{y}_{:,i}|\mathbf{X},\beta)$$
where $\mathbf{y}_{:,i}$ represents the $i^{th}$ column of $\mathbf{Y}$ and
$$p(\mathbf{y}_{:,i}|\mathbf{X},\beta)=\mathcal{N}(\mathbf{y}_{:,i}|\mathbf{0,XX}^T+\beta^{-1}\mathbf{I}).$$

$K_{nn}$ is the $n \times n$ covariance matrix defined by the kernel function $k(\mathbf{x,x'})$.

Since X is a latent variable, we can assign it to a prior density given by the standard normal density. The prior for $X$ is thus
$$p(\mathbf{X})=\prod^n_{i=1}\mathcal{N}(\mathbf{x_i|0, I_Q})$$
where each $\mathbf{x_i}$ is the $i^{th}$ row of $\mathbf{X}$. The joint probability model for the GP-LVM is
$$p(\mathbf{Y,X})=p(\mathbf{Y|X})p(\mathbf{X})$$

Instead of training the GP-LVM model by finding the MAP estimate of $\mathbf{X}$ whilst jointly maximizing with respect to the hyperparameters as introduced in the orginal paper, Bayesian GP-LVM performs variational inference to marginalise the latent variables. This method enables optimisation of the resulting lower bound on the marginal likelihood with respect to the hyperparameters.

\subsection{Variational Inference}
To obtain the marginal likelihood of the observed data, we integrate out the latent variables:
$$p(\mathbf{Y})=\int p(\mathbf{Y|X})p(\mathbf{X})d\mathbf{X}.$$
This method, however, is intractable in practice. Instead, let variational distribution $q(\mathbf{X})$ approximate the posterior distribution over the latent variables, $p(\mathbf{X|Y})$.
$$q(\mathbf{X})=\prod^n_{i=1}\mathcal{N}(\mathbf{x_n|\mu_n,S_n})$$
where the variational parameters are $\mathbf{ \{ \mu_n,S_n \}^n_{i=1} }$ and $\mathbf{S_n}$ is a diagonal covariance matrix. Using this variational distribution, the Jensen's lower bound on $\log p(\mathbf{Y})$ is expressed as
$$F(q)=\int q(\mathbf{X})\log \frac{ p(\mathbf{Y|X})p(\mathbf{X}) }{ q(\mathbf{X}) } d\mathbf{X}$$


\section{Generative Models}

\section{Related Work}
\subsection{Learning a Manifold}
\cite{fontmanifold}


\subsection{Drawing Assistant}

\subsection{AutoHair}
\cite{autohair}
Neural networks
Helicoids
[Automated Production of 3D Assets]
Procedural generation
Generative models
Procedural Generation of Hair 



% -----------------------------------------------------------------------------

\chapter{Project Execution}
\label{chap:execution}

{\bf A topic-specific chapter, of roughly $15$ pages} 
\vspace{1cm} 

\noindent
This chapter is intended to describe what you did: the goal is to explain
the main activity or activities, of any type, which constituted your work 
during the project.  The content is highly topic-specific, but for many 
projects it will make sense to split the chapter into two sections: one 
will discuss the design of something (e.g., some hardware or software, or 
an algorithm, or experiment), including any rationale or decisions made, 
and the other will discuss how this design was realised via some form of 
implementation.  

This is, of course, far from ideal for {\em many} project topics.  Some
situations which clearly require a different approach include:

\begin{itemize}
\item In a project where asymptotic analysis of some algorithm is the goal,
      there is no real ``design and implementation'' in a traditional sense
      even though the activity of analysis is clearly within the remit of
      this chapter.
\item In a project where analysis of some results is as major, or a more
      major goal than the implementation that produced them, it might be
      sensible to merge this chapter with the next one: the main activity 
      is such that discussion of the results cannot be viewed separately.
\end{itemize}

\noindent
Note that it is common to include evidence of ``best practice'' project 
management (e.g., use of version control, choice of programming language 
and so on).  Rather than simply a rote list, make sure any such content 
is useful and/or informative in some way: for example, if there was a 
decision to be made then explain the trade-offs and implications 
involved.

\section{Example Section}

This is an example section; 
the following content is auto-generated dummy text.
\lipsum

\subsection{Example Sub-section}

\begin{figure}[t]
\centering
foo
\caption{This is an example figure.}
\label{fig}
\end{figure}

\begin{table}[t]
\centering
\begin{tabular}{|cc|c|}
\hline
foo      & bar      & baz      \\
\hline
$0     $ & $0     $ & $0     $ \\
$1     $ & $1     $ & $1     $ \\
$\vdots$ & $\vdots$ & $\vdots$ \\
$9     $ & $9     $ & $9     $ \\
\hline
\end{tabular}
\caption{This is an example table.}
\label{tab}
\end{table}

\begin{algorithm}[t]
\For{$i=0$ {\bf upto} $n$}{
  $t_i \leftarrow 0$\;
}
\caption{This is an example algorithm.}
\label{alg}
\end{algorithm}

\begin{lstlisting}[float={t},caption={This is an example listing.},label={lst},language=C]
for( i = 0; i < n; i++ ) {
  t[ i ] = 0;
}
\end{lstlisting}

This is an example sub-section;
the following content is auto-generated dummy text.
Notice the examples in Figure~\ref{fig}, Table~\ref{tab}, Algorithm~\ref{alg}
and Listing~\ref{lst}.
\lipsum

\subsubsection{Example Sub-sub-section}

This is an example sub-sub-section;
the following content is auto-generated dummy text.
\lipsum

\paragraph{Example paragraph.}

This is an example paragraph; note the trailing full-stop in the title,
which is intended to ensure it does not run into the text.

% -----------------------------------------------------------------------------

\chapter{Critical Evaluation}
\label{chap:evaluation}

{\bf A topic-specific chapter, of roughly $15$ pages} 
\vspace{1cm} 

\noindent
This chapter is intended to evaluate what you did.  The content is highly 
topic-specific, but for many projects will have flavours of the following:

\begin{enumerate}
\item functional  testing, including analysis and explanation of failure 
      cases,
\item behavioural testing, often including analysis of any results that 
      draw some form of conclusion wrt. the aims and objectives,
      and
\item evaluation of options and decisions within the project, and/or a
      comparison with alternatives.
\end{enumerate}

\noindent
This chapter often acts to differentiate project quality: even if the work
completed is of a high technical quality, critical yet objective evaluation 
and comparison of the outcomes is crucial.  In essence, the reader wants to
learn something, so the worst examples amount to simple statements of fact 
(e.g., ``graph X shows the result is Y''); the best examples are analytical 
and exploratory (e.g., ``graph X shows the result is Y, which means Z; this 
contradicts [1], which may be because I use a different assumption'').  As 
such, both positive {\em and} negative outcomes are valid {\em if} presented 
in a suitable manner.

% -----------------------------------------------------------------------------

\chapter{Conclusion}
\label{chap:conclusion}

{\bf A compulsory chapter,     of roughly $5$ pages} 
\vspace{1cm} 

\noindent
The concluding chapter of a dissertation is often underutilised because it 
is too often left too close to the deadline: it is important to allocation
enough attention.  Ideally, the chapter will consist of three parts:

\begin{enumerate}
\item (Re)summarise the main contributions and achievements, in essence
      summing up the content.
\item Clearly state the current project status (e.g., ``X is working, Y 
      is not'') and evaluate what has been achieved with respect to the 
      initial aims and objectives (e.g., ``I completed aim X outlined 
      previously, the evidence for this is within Chapter Y'').  There 
      is no problem including aims which were not completed, but it is 
      important to evaluate and/or justify why this is the case.
\item Outline any open problems or future plans.  Rather than treat this
      only as an exercise in what you {\em could} have done given more 
      time, try to focus on any unexplored options or interesting outcomes
      (e.g., ``my experiment for X gave counter-intuitive results, this 
      could be because Y and would form an interesting area for further 
      study'' or ``users found feature Z of my software difficult to use,
      which is obvious in hindsight but not during at design stage; to 
      resolve this, I could clearly apply the technique of Smith [7]'').
\end{enumerate}

% =============================================================================

% Finally, after the main matter, the back matter is specified.  This is
% typically populated with just the bibliography.  LaTeX deals with these
% in one of two ways, namely
%
% - inline, which roughly means the author specifies entries using the 
%   \bibitem macro and typesets them manually, or
% - using BiBTeX, which means entries are contained in a separate file
%   (which is essentially a databased) then inported; this is the 
%   approach used below, with the databased being dissertation.bib.
%
% Either way, the each entry has a key (or identifier) which can be used
% in the main matter to cite it, e.g., \cite{X}, \cite[Chapter 2}{Y}.

\backmatter

\bibliography{dissertation}

% -----------------------------------------------------------------------------

% The dissertation concludes with a set of (optional) appendicies; these are 
% the same as chapters in a sense, but once signaled as being appendicies via
% the associated macro, LaTeX manages them appropriatly.

\appendix

\chapter{An Example Appendix}
\label{appx:example}

Content which is not central to, but may enhance the dissertation can be 
included in one or more appendices; examples include, but are not limited
to

\begin{itemize}
\item lengthy mathematical proofs, numerical or graphical results which 
      are summarised in the main body,
\item sample or example calculations, 
      and
\item results of user studies or questionnaires.
\end{itemize}

\noindent
Note that in line with most research conferences, the marking panel is not
obliged to read such appendices.

% =============================================================================

\end{document}
