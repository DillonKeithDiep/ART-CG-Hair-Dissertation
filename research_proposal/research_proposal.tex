\documentclass[a4paper, fontsize=12pt, onecolumn]{article} % A4 paper and 11pt font size
\usepackage[top=2.5cm, bottom=2.5cm, left=2.5cm, right=2.5cm]{geometry}

\usepackage[pdftex]{graphicx}
\usepackage{hyperref}
\usepackage{multirow}
\usepackage{fourier} % Use the Adobe Utopia font for the document - comment this line to return to the LaTeX default
%\usepackage[english]{babel} % English language/hyphenation
\usepackage{amsmath,amsfonts,amsthm} % Math packages
\usepackage{sectsty} % Allows customizing section commands

\bibliographystyle{plain}

\allsectionsfont{\normalfont\scshape} % Make all sections centered, the default font and small caps

\usepackage{fancyhdr} % Custom headers and footers
%\pagestyle{fancy}
\rhead{ }
\renewcommand{\headrulewidth}{0pt} % Remove header underlines
\renewcommand{\footrulewidth}{0pt} % Remove footer underlines
\setlength{\headheight}{13.6pt} % Customize the height of the header

\numberwithin{equation}{section} % Number equations within sections (i.e. 1.1, 1.2, 2.1, 2.2 instead of 1, 2, 3, 4)
\numberwithin{figure}{section} % Number figures within sections (i.e. 1.1, 1.2, 2.1, 2.2 instead of 1, 2, 3, 4)
\numberwithin{table}{section} % Number tables within sections (i.e. 1.1, 1.2, 2.1, 2.2 instead of 1, 2, 3, 4)

\setlength\parindent{0pt} % Removes all indentation from paragraphs - comment this line for an assignment with lots of text
\setlength\parskip{1em}

\setlength\columnsep{18pt}

\title{
	\vspace{-3.0cm}
	\horrule{0.4pt} \\[0.2cm] % Top horizontal rule
	\vspace{0.2cm}
	\Large ART-CG: Assisted Real-time Content Generation of 3D Geometry through Machine Learning\\
	\horrule{0.4pt} \\[0cm] % Bottom horizontal rule
	\vspace{-0.5cm}
}
\date{} % No date

\newcommand{\horrule}[1]{\rule{\linewidth}{#1}} % Create horizontal rule command with 1 argument of height

%%%%% FOOTNOTE EXAMPLE %%%%%
%\footnote{\href{www.google.com}{www.google.com}}


\begin{document}
\maketitle
\thispagestyle{fancy} % Show header on front page
%\pagestyle{plain} % Remove header on following pages

\section*{The Proposers}
\hrule
\subsection*{The University of Bristol}
\textbf{Dr D. K. Diep} has recently conducted a research project on application of machine learning to hair geometry.

\textbf{Dr C. H. Ek} is a Professor of Computer Science at the University of Bristol. He specialises in machine learning.

\subsection*{ManualDesk}
\textbf{Dr D. Hu} is the research director of ManualDesk.

\subsection*{Circle Enix}
\textbf{Dr C. Strife} is the lead developer of gaming firm Circle Enix.

\newpage

\section*{Case for Support}
\hrule
\section{Overview and Motivation}
\label{motivation}
%%% Background & Hypothesis %%%
The production of 3D assets is a meticulous and time-consuming procedure that also requires expert knowledge. Content developers create, manipulate, and maintain high-dimensional 3D geometric data ranging from an order of thousands to millions of data points. State of the art 3D software offers an extraordinary number of features to tackle the complexity of 3D production. However, this promotes a steep learning curve and builds a high barrier to entry for non-experts to produce 3D content. Our proposition of assisted real-time content generation investigates simplification of 3D asset production through application of machine learning.

%%% National Importance %%%
Application of 3D assets contributes immensely to the general public.
Computer-aided-design (CAD) drives engineering and manufacturing.
Computer-generated imagery (CGI) supports the entertainment industry.
The field of medicine has utilised 3D printing, and its application is projected to expand \cite{3dprinting}.
Visualisation, simulation and research often make use 3D assets \cite{simulation}.
Computer graphics (CG) is not only extensively applied, according to \textit{John Peddie Research (JPR)}, the CG market itself will exceed \$235 billion by 2018 \cite{cgmarket}.

%%% Challenges %%%
Definition of geometric representations is often by describing visual properties. For example, polygonal meshes are formed from \textit{edges}, \textit{vertices},  and \textit{faces}. While such approach is sensible for traditional use, the resulting data will have varying dimensionality. This misalignment of dimensionality is problematic for machine learning methods that require computational evaluation and comparison. Creative content is sensitive to corruption, minor changes to a component can be sufficient to invalidate an entire object as a whole. Thus, conventional feature extraction methods are not applicable to 3D geometric data.

The \textit{topology} of a surface is the arrangement of components. The topological structure has practical consequences on the performance of a 3D Topology determines the rendering efficiency and behaviour in reaction to algorithmic operations. Incorporating the idea of topology creates an ambiguous many-to-one mapping of geometric data to the true surface it represents. This ambiguous aspect of geometric representation again contributes to the difficulty of applying established learning methods.

Traditional learning-based approaches generalise a solution by training with a large data set. Unfortunately, 3D data is not as easily attainable when compared to other mediums such as images. Despite the abundance of exemplary 3D output produced, most are not available to the public. Freely available 3D content vary in stylistic properties and suffer from the lack of quality control. In practice, the pool of available 3D exemplars is small. The learning model applied must be robust with small training sets, imperfect examples, and the presence of uncertainty.

The subjective nature of evaluating artistic products poses a problem for computational solutions. A creative environment demands versatility. This incertitude aspect necessitates learning a regression model that can predict a set of acceptable solutions as an assisting procedure, as opposed to a fully-automated process.

%%% Research hypothesis and objectives %%% 
A learning-based perspective towards production can reuse exemplary data available to automate the task at hand.
Identifying commonly recurring properties remove the need to complete repetitive operations manually, leading to \textit{simplification} of the production pipeline. Non-experts receive the ability to create 3D assets, allowing focus on the human aspect of design without requiring to learn the intrinsics of 3D software. 
A real-time method that uses exemplary data as a reference gives rise to the idea of \textit{rapid prototyping}. Typically, visualising a concept is presented through a medium that is faster to produce, such as conceptual art or blueprint sketches. Such practice is common but risks misrepresentation from translation between mediums. Exposing a fast method of sampling from a latent space learned from exemplary designs can provide initial geometry to be refined and presents a clearer indication of a concept.

Assisted real-time content generation extends the utility of machine learning research for the creative field. Past research has used probabilistic nonlinear dimensionality reduction to learn a \textit{latent space} (low-dimensional representation) from a high-dimensional space, simplifying data by reducing the number of variables needed to be considered. Such an approach has been employed for animation \cite{styleik}, basic drawings \cite{latentdoodle}, and font design \cite{fontmanifold}. ART-CG builds upon the foundation of learning-based solutions for creative purposes established and transfers it to complex, enormously high dimensional 3D geometric data by resolving the challenges outlined.


\section{Programme and Methodology}
\subsection*{Work Package 1 (WP1): Parametrisation of 3D Objects}
\textbf{Research Leader:} Dr D. K. Diep\par

The principal research objective of this work package is to establish an intermediate standardised 3D representation suitable for machine learning. As outlined in section \ref{motivation}, conventional feature extraction methods are insufficient to represent delicate details of surfaces. WP1 investigates methods of approximating object successfully so that the aligned data are comparable and minimises the data loss from approximation.

Geometric representations can describe arbitrary surfaces, adjusting to the need of a content creator flexibly. Unusual surface output is undesirable for the purpose of assisted content generation. Constraints can be placed on the space of possible surfaces described to ensure a level of quality. In computer graphics, there exists procedural generative methods that synthesise complex geometry from a smaller set of input parameters.

The first objective of this phase is to evaluate current generative models available as candidates that can be intermediate representations for machine learning methods. The industrial standard for CGI predominately uses mesh models, a high-fidelity representation but is susceptible to noise corruption - marginal alteration on the vertices of a mesh model can change its appearance significantly. Generative models and representations of surfaces is a large sub-topic with much to explore. Potential intermediate representations include generalised cylinders, voxels, shape histograms, and 3D salient points \cite{salientpoints}. It is critical for the generalised model to strike a balance between the cardinality of inputs required and the fidelity it is able to capture.

From the set of existing models, we will formulate an extended model such that it contains a comparable metric and preserves the necessary properties of geometric data including preservation of the topology. Finally, we devise sampling methods to acquire an approximation of a 3D asset by accurately estimating the inputs of a generative model.

\subsubsection*{Deliverables} 
\begin{enumerate}
	\item \textbf{D1a.} Investigate techniques or generative model(s) that is suitable for resolving the alignment problem of raw input data, allowing a sufficient approximation of generative parameters with the same dimension.
	\item \textbf{D1b.} Formulate techniques or generative model(s) that preserve geometric properties to maintain an organised topology desirable for 3D assets.
	\item \textbf{D1c.} Devise methods to sample generative inputs from approximating a 3D asset to the closest equivalent defined in a generative model.
\end{enumerate}

\subsection*{Work Package 2 (WP2): Learning a Regression Model of 3D Objects}
\textbf{Research Leader:} Dr C. H. Ek\par

The research objective of this phase is to investigate methods that are ideal for learning the generation of 3D objects. There are many machine learning models that can be used for regression. A model that is fitting for production of 3D assets must address the following:
\begin{itemize}
	\item Subjective evaluation of artistic products.
	\item Small training data sets.
	\item Missing information and uncertainty.
\end{itemize}

Past research have utilised the Gaussian process latent variable model (GP-LVM) to learn a probabilistic nonlinear latent manifold embedding of the observed data. The GP-LVM model is a strong contender as a regression model for the ART-CG framework as a probabilistic solution can offer multiple acceptable predictions for a content creator to choose from \cite{gplvm}. Gaussian processes are non-parametric, thus, it can flexibly capture a range of possible values fitting for general 3D assets. Chai et al (2016) employed a neural network as part of a system that produces 3D hair geometry \cite{autohair}. The scope of investigation encapsulates comparison of different regression models available.

The complexity of 3D geometry calls for advanced regression models. Viable models are to be extended with concepts such as layering hierarchy or applying deep learning \cite{deepgp}. Considerations include the limitations of learning models, evaluating the trend of performance as training data set scales. Online (continually) learning models update with new observations without requiring retraining, this approach is a possible solution for refining regression models as 3D asset is produced even with an initially small training set.

\subsubsection*{Deliverables} 
\begin{enumerate}
	\item \textbf{D2a.} Investigate viable learning models that are suitable for generation of 3D objects.
	\item \textbf{D2b.} Applying research of complex models such as hierarchical layering, deep learning, and online training to utilise an advanced regression model for complex 3D geometry.
	\item \textbf{D2c.} Assorting competitive kernels that establish effective default prior probability distributions suitable for 3D geometry regression models and learning hyperparameters.
\end{enumerate}

\subsection*{Work Package 3 (WP3): Generation and Reparation}
\textbf{Research Leader:} Dr D. K. Diep\par

Prediction by regression models contain a level of uncertainty. In the context of 3D assets, there are scenarios where specific constraints for geometry is outlined. An example would be hair object that should not grow inwards into a scalp surface. Engineering design and architecture also apply strict limitations. This phase investigates methods of generation that conform with specified constraints, performing reparations appropriately so that the output is optimised and practical. 

Constraints can be applied within the regression model through weighting inputs or implementing a cost function that penalises undesirable attributes.  Post-prediction reparation would consider regulating output through heuristics or physical simulation to remove overlapping surfaces and similar properties.

\textbf{Deliverables}
\begin{enumerate}
	\item \textbf{D3a.} Investigate and formulate methods that constrain the regression model.
	\item \textbf{D3b.} Investigate and formulate methods that repair the output of a regression model.
\end{enumerate}

\subsection*{Work Package 4 (WP4): High-Fidelity Generalised Parametrisation}
\textbf{Research Leader:} Dr C. Strife\par

Techniques such as vector displacement maps provide an additional layer of detail for 3D assets. This phase looks into incorporating displacement maps and similar methodologies as a part of the ART-CG framework to improve the quality of predictions by the regression model without sacrificing topology.

\subsubsection*{Deliverables} 
\begin{enumerate}
	\item \textbf{D4a.} Learning an additional model of high-fidelity generalised parametrisation to modify 3D assets.
	\item \textbf{D4b.} Incorporating high-fidelity generalised parametrisation as part of the generative model for the 3D geometry regression model.
\end{enumerate}


\subsection*{Work Package 5 (WP5): Intuitive Presentation of Latent Control Variables}
\textbf{Research Leader:} Dr D. Hu\par
Simplifying the production process through learning a regression model delegates control from the observed variables to descriptive latent (unobserved) variables that are input to the regression model. Determining the output with latent variables reduce the number of inputs required, but serves no purpose if the latent variables are counter-intuitive. This phase aims to deliver an intuitive presentation for latent control variables.

Discover factors that determine performance attained and identify the performance differences of scenarios, such as real-time programs.
When applicable to latent variables, determining whether it is possible to infer semantic meaning from common latent variables for classification.

\subsubsection*{Deliverables} 
\begin{enumerate}
	\item \textbf{D5a.} Evaluating and devising methods of associating latent control variables to semantic properties.
	\item \textbf{D5a.} Presenting latent control variables intuitively.
	
\end{enumerate}

\newpage

\bibliography{research_proposal}

\newpage

\section*{Budget}
\hrule

\begin{table}[!h]
	\centering
	{\renewcommand{\arraystretch}{1.8} %<- modify value to alter spacing
	\begin{tabular}{|l|c|c|c|}
		\hline
		Item		& Quantity	& Annual cost per unit (\textsterling)	& Total cost over duration (\textsterling)\\
		\hline
		PI			& 1			& 100,000				& 300,000\\
		\hline
		CoI			& 1			& 75,000				& 210,000\\
		\hline
		PD			& 2			& 50,000				& 300,000\\
		\hline
		PhD			& 2			& 20,000				& 120,000\\
		\hline
		Technical Artist (10\%) & 1 & 10,000			& 30,000\\
		\hline
		Workstation	& 2			& 500					& 1,000\\		
		\hline
		Workshop	& 2			& 1,000					& 2,000\\		
		\hline
		Conference	& 3			& 6,000					& 18,000\\		
		\hline
		Travel \& accommodation & X	& 5,000				& 15,000\\		
		\hline
		Training Data& X		& X						& 150\\
		\hline
	\end{tabular}
	}
	%\caption{budget}
\end{table}

\vspace{1cm}

\textbf{Total cost:} \textsterling 996,150.00

\newpage

\section*{Justification for Resources}
\hrule

The principal investigator (PI) and co-investigator (CoI) oversees scheduling, administration, and execution of the research project such that it successfully achieves the deliverables promised.
The principal investigator, Dr D. Diep, is the instigator of the research proposal and will be the one with ultimate responsibility for directing the project in the right direction.
The co-investigator, Dr. C. H. Ek, will support and assist the principal instigator in leading the research project.
The investigators will ensure that the project is executed in compliance with governing laws and regulations, as well as adhering to the policies outlined by sponsors.

Post-doctoral researchers bring expertise knowledge from related fields that will assist in making informed decisions to carry out research efficiently. PhD students will support the development of the research project and acquire skills from the field of machine learning, such as probabilistic non-linear dimensionality reduction and other significant learning-based models applicable to the 3D production pipeline.

A part-time technical artist will create and modify geometric data to incorporate extreme or specific test cases considered by the project. An initial training set is purchased as it is more efficient than commissioning a professional artist. However, as the project develops, the input of a professional artist will be required to successfully introduce ART-CG to the production pipeline.

Circle Enix and ManualLab will contribute with their industrial expertise for this project. Exchanging ideas create synergy between academia and industry to improve the chance of delivering successful results.

Two shared workstations in office will process the data and prototyping ideas. While each member of staff is expected to own a personal machine, the workstations will act as a central point for development. Circle Enix has agreed to offer their cloud computing resources as well as access to their render farm facility for this project.

The purpose of conferences and communication is to stay up to date with latest state of the art research and spreading awareness of our research project. National and international travel and accommodation costs is included in the budget. Publicity for the initiative will encourage adoption and draw in potential opportunities for collaboration. Exchanging findings with other researchers will be beneficial for all projects within the community.

\newpage

\section*{Pathways to Impact}
\hrule
As mentioned in section \ref{motivation}, the production of 3D assets play a pivotal role in business, engineering, and entertainment. It is an industry predicted to be worth \$ 235 billion in 2018. The collaborative effort between University of Bristol with industrial leaders Circle Enix and ManualLab will give an excellent chance of formulating effective methods/

More 3D in future, Microsoft 3D Paint, augmented reality on tablets and smartphones, etc

Related topics: Graphics and visualisation, artificial intelligence, machine-learning

Dimensionality reduction on enormously high dimensional data.

Applications that allow users to create their personal content could also integrate machine-learning based tools to prevent inappropriate or undesirable creation from being produced while providing options that surpass existing alternatives. An example would be avatar creation for many applications and video games. A space of reasonable options generated from predefined outputs by the developers will allow users to interpolate between sensible configurations, providing an excellent level of customisation while adhering to defined constraints.

The research department of ManualDesk is interested in methods of effectively presenting a low-dimensional representation acquired from dimensionality reduction such that content creators would find intuitive to use.

Circle Enix is seeking assisted content creation tools for non-expert consumers to create custom assets for their upcoming video game software.

\newpage

\section*{Work Plan}
\hrule

\end{document}