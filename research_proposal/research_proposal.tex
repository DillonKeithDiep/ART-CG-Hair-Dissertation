\documentclass[a4paper, fontsize=15pt, onecolumn]{article} % A4 paper and 11pt font size
%\documentclass[a4paper,11pt]{article}
%\usepackage[a4paper, total={7.5in, 9.5in}]{geometry}
\usepackage[top=2.5cm, bottom=2.5cm, left=2.5cm, right=2.5cm]{geometry}

\usepackage[pdftex]{graphicx}
\usepackage{hyperref}
\usepackage{multirow}
\usepackage{fourier} % Use the Adobe Utopia font for the document - comment this line to return to the LaTeX default
%\usepackage[english]{babel} % English language/hyphenation
\usepackage{amsmath,amsfonts,amsthm} % Math packages
\usepackage{sectsty} % Allows customizing section commands

\bibliographystyle{plain}

\allsectionsfont{\normalfont\scshape} % Make all sections centered, the default font and small caps

\usepackage{fancyhdr} % Custom headers and footers
%\pagestyle{fancy}
\rhead{ }
\renewcommand{\headrulewidth}{0pt} % Remove header underlines
\renewcommand{\footrulewidth}{0pt} % Remove footer underlines
\setlength{\headheight}{13.6pt} % Customize the height of the header

\numberwithin{equation}{section} % Number equations within sections (i.e. 1.1, 1.2, 2.1, 2.2 instead of 1, 2, 3, 4)
\numberwithin{figure}{section} % Number figures within sections (i.e. 1.1, 1.2, 2.1, 2.2 instead of 1, 2, 3, 4)
\numberwithin{table}{section} % Number tables within sections (i.e. 1.1, 1.2, 2.1, 2.2 instead of 1, 2, 3, 4)

\setlength\parindent{0pt} % Removes all indentation from paragraphs - comment this line for an assignment with lots of text
\setlength\parskip{1em}

\setlength\columnsep{18pt}

\title{
	\vspace{-3.0cm}
	\horrule{0.4pt} \\[0.2cm] % Top horizontal rule
	\vspace{0.2cm}
	\Large ART-CG: Assisted Real-time Content Generation of 3D Geometry through Machine Learning\\
	\horrule{0.4pt} \\[0cm] % Bottom horizontal rule
	\vspace{-0.5cm}
}
\date{} % No date

\newcommand{\horrule}[1]{\rule{\linewidth}{#1}} % Create horizontal rule command with 1 argument of height

%%%%% FOOTNOTE EXAMPLE %%%%%
%\footnote{\href{www.google.com}{www.google.com}}


\begin{document}
\maketitle
\thispagestyle{fancy} % Show header on front page
%\pagestyle{plain} % Remove header on following pages

\section*{The Proposers}
\hrule
\subsection*{The University of Bristol}
\textbf{Dr D. K. Diep} has recently conducted a research project on application of machine learning to hair geometry.

\textbf{Dr C. H. Ek} is a Professor of Computer Science at the University of Bristol. He specialises in machine learning.

\subsection*{ManualDesk}
\textbf{Dr D. Hu} is the research director of ManualDesk.

\subsection*{Circle Enix}
\textbf{Dr C. Strife} is the lead developer of gaming firm Circle Enix.

\newpage

\section*{Case for Support}
\hrule
\section{Overview and Motivation}
%%% Intro %%%
The production of 3D assets is a meticulous and time-consuming procedure that also requires expert knowledge. Content developers create, manipulate, and maintain high-dimensional 3D geometric data ranging from order of thousands to millions of data points. State of the art 3D software offer an extraordinary number of features to tackle the complexity of 3D production. However, this promotes a steep learning curve and builds a high barrier to entry for non-experts to produce 3D content. Our proposition of assisted real-time content generation investigates simplification of 3D asset production through application of machine learning.

%%% Importance of 3D modelling %%%
Application of 3D assets contribute immensely to the general public.
Computer aided-design (CAD) drives engineering and manufacturing.
Computer-generated imagery (CGI) supports the entertainment industry.
3D printing has been utilised in the field of medicine and its application is projected to expand \cite{3dprinting}.
Visualisation, simulation, and research often make use 3D assets \cite{simulation}.
Computer graphics (CG) is not only extensively applied, according to \textit{John Peddie Research (JPR)}, the CG market itself will exceed \$235 billion by 2018 \cite{cgmarket}.

%%% About our proposal %%%
A learning-based perspective towards production can reuse exemplary data available to automate the task at hand.
Identifying commonly recurring properties remove the need to manually complete repetitive operations. This leads to \textit{simplification} of the production pipeline. Non-experts receive the ability to create 3D assets, allowing focus on the human aspect of design without requiring to learn the intrinsics of 3D software. 
A real-time method that uses exemplary data as reference gives rise to the idea of \textit{rapid prototyping}. Typically, visualising a concept is presented through a medium that is faster to produce, such as conceptual art or blueprint sketches. Such practice is common but risks misrepresentation from translation between mediums. Exposing a fast method of sampling from a space that is learned from exemplary designs can provide initial geometry to be refined and provides a clearer indication of the concept. Assisted content generation is useful for professional content creators whom operate in an environment where division of labour is paramount.

\section{Challenges}
The role that learning methods play in both manufacturing and consumer application continue to grow. However, adoption has been slow for creative fields. Applying machine learning to 3D production faces multiple novel challenges. 

Geometric data represent a surface through components and come in varying dimension. Most often, the arrangement of components also come into factor, such as topology of polygonal meshes. This creates a many-to-one mapping of geometric data to the true surface it represents. Creative content is sensitive to corruption, small changes to a component can invalidate an entire object as a whole. Thus, typical feature extraction methods are not applicable.

Machine learning solutions quite often use large training sets. The problem with 3D data is that it is not as easily attainable when compared to other mediums such as images. There are many freely available, but quality is not the best. In practice, the pool of available 3D exemplars is small.

The ambiguous nature of evaluating art poses a problem. A creative environment demands versatility. This is problematic for machine learning methods.

Real-time performance is necessary to provide immediate feedback as artists won't wait idly.

\section{Academic Background}

% Style-based inverse kinematics introduced the Scaled Gaussian Process Latent Variable Model to learn the probabilistic distribution of a 3D human posture model [15]. Motion capture data represent character posture with a 42-dimensional feature vector that encapsulates joint information of a humanoid body. Learning a model of poses established the relation between joints and identified constraints exhibited in the training data. The probabilistic distribution is plot and used to select predictions in the model. There are no constraints on the joints, but sampling areas of high likelihood in the distribution models realistic motion that resembles the input data.

% A latent doodle space learns latent (unobserved) variables that describe simple line drawings more concisely than the original data [7]. The motivation of a latent doodle space is to generating new drawings that are inspired by the input data. There are two key phases to derive a latent doodle space: the first challenge is to identify line strokes within drawings, the second is using a latent variable method to learn a low-dimensional latent space of input drawings. 

% A study by Campbell and Kautz (2014) presented a framework that learns the latent manifold of existing font styles [10]. The process involved universal parametrization of fonts to a polyline representation so that a distance measure is applicable and the generative model can interpolate between styles. Non-experts could create font styles without experience on type design by sampling points from a two-dimensional latent manifold.

% Drawing assistance powered by large-scale crowd-sourcing explored the potential of data-driven drawing to prompt for correction by achieving an artistic consensus [20]. Learning a correction vector field from training drawings finds a consensus. Stroke-correction is applied using the correction vector field to adjust user input dynamically.

% Chai et al. (2016) introduced AutoHair, a method for automatic modelling of 3D hair from a portrait image [12]. The approach extracts information from images and uses a database of hair meshes to construct a 3D representation of the information conveyed. A hierarchical deep neural network trained on annotated hair images learn to segment hair and estimate growth direction within portraits. Data-driven hair matching and modelling algorithm fit meshes from the database to parameters output by the neural net model to automatically produce 3D hair. The experiment developed a traversable hairstyle space of 50,000 hair models, using training images to fit segments of 3D examples obtained from the internet.

Research regarding creative content often parametrise the training data so that machine learning is applicable. There is no clearly defined solution for a problem in the creative field, effective solutions strive for versatility, employing consensus decision making or offering multiple solutions. To overcome the challenges introduced, dimensionality reduction through unsupervised learning with probabilistic latent variable models such as the Gaussian Process Latent Variable Model (GP-LVM) [18] present an opportunity to learn stylistic properties of design and predict multiple acceptable outputs by analysing the likelihood.


\section{Programme and Methodology}
\subsection{Work package 1 (WP1): Parametrisation of 3D Objects}
A critical problem 

3D objects are defined ambiguously, it is challenging to compare structures of 3D geometry without preprocessing. This phase investigates methods of approximating object successfully so that the aligned data are comparable and minimises the data loss from approximation.

There are numerous representations of 3D objects in computer graphics. One way to obtain 3D geometry data is to sample surfaces of the physical world with a 3D scanner. Common representations of sampled geometric data include point clouds, range maps, and voxels (figure 1.1).

3D surface representations have advantages and disadvantages depending on the use case. It is possible to convert between representations; however, conversion between representations risk incurring data loss. The CAD (Computer-Aided Design) industry often uses precise mathematical representations such as NURBS (Non-Uniform Rational Basis Spline). The most widely adopted representation for CGI (Computer-Generated Imagery) is polygonal meshes. In a production environment, it is preferred to define geometry specifically to requirements of the design as opposed to capturing examples. Polygon meshes are simple to define, yet with established techniques such as UV texturing and normal mapping, are sufficiently expressive for visualising purposes. The study of polygonal meshes is deeply rooted in computer graphics.

Elements of a polygonal mesh are vertices, edges, and faces. The topology of a mesh is concerned with the arrangement of its components, well-organised topology is required to maintain geometric qualities
when performing algorithmic operations on a mesh. In practice, professionals create polygon meshes with majority quad-face topology (faces formed from four edges) during production. The rendering pipeline often automatically converts polygon meshes to triangle faces (formed from three edges) as an optimisation process. Quadrilateral mesh form what we call edge loops which can be used to define the structure of geometry, thus conform better with editing tools and preserve structure when algorithmically processed in comparison to triangle meshes.

The preferred representation of 3D objects within the industry is polygons. A polygonal object is defined by vertices, edges, and faces that form the geometry. Topology is the study of geometrical arrangement for the object. It is best practice to use quad-faces, which are faces formed from four edges asset production. The rendering pipeline often requires and automatically converts geometry to tri-faces, faces formed from three edges. Other representations include voxels which represent objects with cubes, and pixols, a 3D variant of pixels introduced by Pixelogic for their state of the art sculpting program, ZBrush.


\subsection{Deliverables}
\begin{enumerate}
	\item Investigate and formulate techniques or generative model(s) that resolve the alignment problem of raw input data, allowing a sufficient approximation of generative parameters with the same dimension.
	\item Implementation of said approximators, integrated within the production pipeline (extending state of the art tools such as 3DS Max, Maya, and Blender).
\end{enumerate}

A challenge presented when attempting to learn the variation of 3D objects is the alignment problem. Data produced by creators defined in the form of vertices, edges and faces have varying dimensionality. Representations of the same object can vary, due to the level of detail and the pose of the object. Approximation of the object should be: translation, rotation, scale, and level of detail invariant, demanding a volumetric or an alternative generic approach.

Voxels have had success in estimating 3D objects but typically require closed shapes, this is not enough for production as planes, and unfilled shapes are often used to optimise by reducing the polycount.

\subsection{Work Package 2 (WP2): Learning the Generation of 3D Objects}
%%% Automated methods, procedural generation %%%
Procedural generation techniques can automatically produce output that adheres to rules established by the generative model defined. Generation of terrains and city modelling sometimes employ procedural techniques to take advantage of its systematic nature to mass produce variations in agreement with specified patterns [11]. Fractals and methods such as the Lindenmayer system have been used to create patterns that resemble those observed in nature [24]. Automated techniques such as the ones discussed, however, are seldom used for modelling distinct objects with a specific design. It is an involved process to control the output of procedurally generated content without heavily restricting its capabilities. Automated methods that do not learn cannot adapt to changing demands without reimplementation, motivating for a learning-based solution.

Existing tools for 3D modelling have remained mostly static in the paradigm of approach over the past several decades. Automation through methods such as procedural generation can produce content faster. However, traditional synthesis and automated solutions are defined to produce a specific class of output through established patterns and cannot adapt to new models without reimplementation. The research hypothesis of this study is that of applying nonlinear probabilistic dimensionality reduction improves the efficacy of creative content production for exceptionally high dimensional data such as complex 3D hair geometry on virtual humanoids.


Research Objective: Investigate methods that are ideal for learning the generation of 3D objects and how to best visualise it so that it is useful. The scope of investigation encapsulates comparison of different regression models, extending viable models through layered hierarchy or applying deep learning. When applicable to latent variables, determining whether it is possible to infer semantic meaning from common latent variables for classification. Considerations include the limitations of learning models, evaluating the trend of performance as training data set scales.

\subsubsection{Deliverables} 
\begin{enumerate}
	\item Investigate viable learning models that are suitable for generation of 3D objects.
	\item Critically evaluating the advantages and disadvantages of each model.
	\item Applying research of complex models such as layered models and deep learning.
\end{enumerate}

\subsection{Work Package 3 (WP3): Generation and Reparation}

Research Objective: The output of trained models contain a level of uncertainty. In the context of 3D assets, it is often flexible with approximation, but some scenarios outline specific constraints. An example would be hair object that should not grow inwards into a head object. Engineering design and architecture also apply strict limitations. This phase investigates methods of generation that conform with specified constraints, performing reparations appropriately so that the output is optimised and practical.

\subsubsection{Deliverables}
\begin{enumerate}
	\item Investigate and formulate methods of generation and reparation for the uncertain output of a trained model.
\end{enumerate}

\subsection{Work Package (WP4): Integration and Applications of Machine-Trained Tools}
Research Objective: Present a framework or specification of training-based tools that are fitting of integration to production and consumer applications. Discover factors that determine performance attained and identify the performance differences of scenarios, such as real-time programs.

\subsubsection{Deliverables}
\begin{enumerate}
	\item Evaluation of online (continually) learning models against offline learning models.
	\item Developing a framework of ideal workflow that integrates machine-trained tools into state of the art production software.
	\item Measure the efficacy for both expert and consumer use cases.
\end{enumerate}
	
\bibliography{research_proposal}

\newpage

\section*{Budget}
\hrule
Employment
Machines \& monitors, accessories
Travelling
Conference
Communication

\newpage

\section*{Justification for Resources}
\hrule
\newpage

\section*{Impact Statement}
\hrule
The production of 3D assets play a pivotal role in business, engineering, and entertainment.
[Cite market surveys of CAD, business, visualisation, entertainment, other benefits]

More 3D in future, Microsoft 3D Paint, augmented reality on tablets and smartphones, etc

Related topics: Graphics and visualisation, artificial intelligence, machine-learning

Dimensionality reduction on enormously high dimensional data.

Applications that allow users to create their personal content could also integrate machine-learning based tools to prevent inappropriate or undesirable creation from being produced while providing options that surpass existing alternatives. An example would be avatar creation for many applications and video games. A space of reasonable options generated from predefined outputs by the developers will allow users to interpolate between sensible configurations, providing an excellent level of customisation while adhering to defined constraints.


\newpage

\section*{Work Plan}
\hrule

\end{document}